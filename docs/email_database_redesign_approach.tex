\documentclass[11pt]{article}
\usepackage[utf8]{inputenc}
\usepackage{geometry}
\usepackage{parskip}
\geometry{margin=1in}

\title{Email: Proposed Approach for Database Redesign}
\author{}
\date{}

\begin{document}
\maketitle
\thispagestyle{empty}

\noindent
\textbf{To:} [Recipient] \\
\textbf{From:} [Your name] \\
\textbf{Subject:} Proposed approach for the database redesign

\vspace{1em}
\noindent
Hi [Name],

\vspace{0.5em}
\noindent
I’ve been thinking about how we can run the database redesign in a way that keeps everyone aligned and reduces rework. Below is a proposed approach and what I’d need from the team to make it work.

\vspace{0.75em}
\noindent
\textbf{Proposed approach}

\vspace{0.5em}
\noindent
\textit{1. Discovery \& documentation} \\
Map and document the current schema (tables, relationships, key constraints) and the target model we want. Having a single source of truth—with diagrams where helpful—will make it easier to spot gaps and get sign-off before we touch production.

\vspace{0.5em}
\noindent
\textit{2. Schema \& data model design} \\
Define the new entities, relationships, and naming in a structured way. I’ll draft the target schema and share it for review so we can agree on it before implementation and migration planning.

\vspace{0.5em}
\noindent
\textit{3. Validation with simple prototypes} \\
For the most important areas, I’ll put together lightweight prototypes (e.g. simple UIs or flows that assume the new schema) so we can validate workflows and requirements with stakeholders before we lock the design. This should reduce the risk of building the wrong thing.

\vspace{0.5em}
\noindent
\textit{4. Migration \& implementation} \\
Once the target design is agreed, we proceed with migration strategy, scripts, and rollout as we’ve discussed, with the documentation and prototypes as reference.

\vspace{0.75em}
\noindent
\textbf{What I need from your side}

\begin{itemize}
  \item Confirmation of priorities (which domains or tables we tackle first).
  \item Availability for short review sessions when I have draft schemas or prototypes to show.
  \item Any hard constraints (e.g. compliance, reporting, integrations) so they’re baked into the design from the start.
\end{itemize}

\vspace{0.75em}
\noindent
\textbf{What this gives us}

\vspace{0.5em}
\noindent
Clear documentation, fewer surprises later, and a way to validate the new design with non-technical stakeholders before we commit. If this approach works for you, I’ll start with step 1 and loop you in as soon as there’s something concrete to review.

\vspace{1em}
\noindent
Best regards, \\
[Your name]
\end{document}
